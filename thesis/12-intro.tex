\Introduction
Все большую популярность в мире приобретают решения по управлению современными устройствами, окружающими человека в быту, получившие общее наименование ''умный дом'' (''smart home''). Точных определений того, что понимается под ''умными'' алгоритмами и ''умным домом''  до настоящего времени нет. Каждый год проходит немало конференций, на которых различные команды излагают свои подходы с данной теме (см., например, \cite{Raisul2012}).

Опережающее развитие в рамках концепции ''умный дом'' получили решения в области управления энергоэффективностью и освещением.  При этом появление современных светодиодов с высокой яркостью свечения сделало решения по управлению освещением современного дома доступными и по-настоящему умными (см.,например \cite{Kumar2017}). 

В 2017 году командой ООО ''Лаборатория моделирования систем'' был  предложен  способ дистанционного управления электронными устройствами ''умного дома'' путём распознавания жестов, выполняемых специальным устройством - ''волшебной палочкой'' \cite{belyaikina2017}.

Само устройство помещается в руке и представляет собой пластиковый цилиндрический корпус, внутри которого находятся источник питания, микропроцессор, модуль инерциальной навигации, устройство Bluetooth связи, направленный приемо-передатчик оптических ИК сигналов (фотоприёмник и светодиод).

Устройство работает следующим образом: при нажатой кнопке включения активируются сенсоры устройства. 

Выходные данные с гироскопов, магнитометров,  акселерометров и фотоприемника подаются для обработки  на вход вычислительного устройства.  

На основе обработки сигналов с сенсоров производится оценка девятимерного вектора движения палочки, как функции времени.  
Далее выполненное движение классифицируется путём  формирования точки в N-мерном пространстве признаков.

Таким образом, устройство позволяет переводить движение, которое пользователь выполняет палочкой в некоторый символ специального алфавита - «словаря жестов». 
Связывая символы данного словаря и команды управления, мы получаем инструмент  управления «умными игрушками» или бытовой электроникой.

Аналогичные устройства управления уже существуют на рынке. 
Но все они являются лишь периферией в некоторых системах (Wii Remote, Microsoft Kinect). 

В рамках представленной работы была поставлена задача интегрировать в рамках единой концепции и архитектуры описанное устройство, мобильное устройство - смартфон и несколько устройств ''умного освещения''.

В рамках дипломной работы решены следующие задачи:

\begin{itemize}
	\item изучено текущее состояние уровня техники в области ''умного освещения'';  
    \item разработана система команд для управления ''умной гирляндой'', построенной на адресуемых RGB светодиодах; 
    \item разработана архитектура взаимодействия  ''умной гирлянды'', описанного выше устройства жестового управления ''умным домом'' и мобильного приложения;
    \item  программная реализация мобильного приложения для управления ''умным домом'';
    \item  разработана заявка на патент на ''Способ управления освещением и устройство для его реализации''.
\end{itemize}

 
В первом разделе работы содержится постановка задачи диплома и подзадач, на которые она была разбита.

Во втором разделе работы приведен обзор научно - технической литературы по теме исследования.

В четвертом разделе описывается подход к решению задачи.

В пятом разделе приведены описания проведённых в ходе работы экспериментов.

  

Работа выполнена в рамках научно-исследовательской работы ООО <<Лаборатории моделирования систем>> по разработке устройства <<Magget>> жестового управления электроникой бытового и игрового назначения и его интеграции в инфраструктуру ''умного дома''.