\pagestyle{empty} % нумерация выкл.
\chapter{ОТЗЫВ}
\textbf{на работу студента СОЛОВЬЕВА Е.А.}

Соловьев Е.А. проходил дипломную практику в ООО “Лаборатория моделирования систем” с сентября 2019 года.  

Тема дипломной работы - «Разработка технологии управления распределенной системой “умного освещения” с элементами искусственного интеллекта» связана с решением задачи построения систем класса smart home, при построении которых используется широкий спектр устройств (исполнительных, интерфейсных, инфраструктурных), платформ для их взаимодействия и приложений. Егору была поставлена задача сопряжения манипулятора “волшебная палочка” и устройства “новогодняя гирлянда” с помощью приложения на мобильном телефоне. 

Егор является основным исполнителем по данной теме и сумел освоить как аппаратную часть, так и осуществить программную реализацию в виде прототипа приложения для устройств iOS.  

Все поставленные в рамках работ задачи успешно выполнены. Уровень выполненной работы и имеющийся план дальнейшего развития проекта позволяют сформулировать задачу для написания магистерской диссертации.  

За время работы Егором разработан прототип приложения для мобильных телефонов iOS, позволяющий сопрягать манипулятор “волшебная палочка” с исполнительными устройствами.  

Работа предоставлена в срок.  

К недостаткам работы можно отнести отсутствие анализа при выборе архитектур примененных в работе.  

На основании вышеизложенного рекомендую оценить работу оцен кой «отлично», а СОЛОВЬЕВА Егора Александровича рекомендовать к поступлению в магистратуру МФТИ.
	
 \begin{tabular}{p{200pt}p{100pt}p{100pt}} \\[10pt]
        Научный руководитель доцент & &~А.В.Хельвас\\  
    \end{tabular}    